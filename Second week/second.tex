\documentclass[12pt letterpaper]{article}

% Use Packages ==================================
\usepackage{hyperref} \setcounter{tocdepth}{2} %for referencing the sections
\usepackage[a4paper, hmargin= 1in, vmargin= 1.5in]{geometry} %for page shaping
\usepackage{fancyhdr} %for Page Styling
\usepackage{multicol} % for Multicolumn generation
\usepackage{lipsum} % Generate Lorem Ipsum Text
\usepackage{xcolor} % For coloring vrules
\usepackage{parskip}[parfill]
\usepackage{ragged2e} 

% Decide Page Style ==============================
\pagestyle{fancy} % Declares that a custom style will be used
\fancyhf{} %Removes the default header and footer

\lhead{Second Week}
\rhead{Section \thesection} %\thesection = section number
\lfoot{Prakhar Mittal}
\rfoot{Page \thepage}
\chead{\leftmark} %Leftmark uses the name of the topmost layer (section with depth 0)
\cfoot{Something else}
\renewcommand{\footrulewidth}{1pt}

% Create Vertical Ruling for Columns =====================
\setlength{\columnseprule}{1pt}
\def\columnseprulecolor{\color{blue}}

% Set Title =======================================
\title{Second Week}
\author{Prakhar Mittal}

% Make Index ========================================
\usepackage{imakeidx}
\makeindex[columns=1, title=My Index, intoc]

% Begin ==========================================
\begin{document}

\maketitle
\tableofcontents
\thispagestyle{empty} %removes every kind of page styling

\clearpage
\pagenumbering{arabic} % To start numbering from this page instead of contents page

\section{Referencing and Table Of Contents}
\label{first}
    \subsection{This is a Sub-Section}
    \label{subsection}
    This subsection is a part of Section \ref{first}.
        \subsubsection{One Point One}
        \label {1.1}
        Testing, 1,2,3.
        
        \addtocontents{toc}
        {\setcounter{tocdepth}{3}}

        \subsubsection{One Point Two}
        \label{1.2}
        Just for testing Purposes.

\addtocontents{toc}
{\setcounter{tocdepth}{2}}
\newgeometry{left=1in, top=1.2in, right=1in, bottom=1.5in} % \clearpage is included

\section{Page Layout}
The global values are first set in the preamble itself using the "geometry" tag. 
To style a specific page, the "newgeometry" and "restoregeometry" tags are used. 
Change the values in the tags to see the difference in page alignment. 
The paper size, type, and orientation cannot be changed however using the newgeometry tag.

\restoregeometry %clearpage is included

\section{Headers, Footers and Footnotes}
    \subsection{Page Styling}
    This is done in the preamble. 
    It creates the fancy header and footer you see at the top/bottom of the page. 
    The default style (plain) is no header and page-number in the center of footer.
    We can set commands so that certain pages are not included in the styling (such as 
    the contents page in this document). 

    \subsection{Footnotes}
    Footnotes are those small comments that are given in a document\footnote{Like this one.} at the bottom
    of the page\footnotemark[2] for clarification and stuff and help in not breaking the flow of the 
    reader.
    \footnotetext[2]{Or this one.}

    For reusing the previously used footnotes, we can use\footnotemark[\value{footnote}]
    
    \renewcommand{\thefootnote}{\fnsymbol{footnote}}
    We can also change the style of footnotes used like \footnotemark[3] or \footnotemark[4].
    This was done using renewcommand on "thefootnote", there are several other styles available, such as
    \begin{itemize}
        \item {\emph{arabic} Arabic numerals.}
        \item {\emph{Roman} Roman numerals.}
        \item {\emph{alph} Alphabetic lower case.}
        \item {\emph{Alph} Alphabetic upper case.}
        \item {\emph{fnsymbol} A special set of 9 characters.}
    \end{itemize}

    \footnotetext[3]{Like this one}
    \footnotetext[4]{Or this one}

\clearpage

\section{Indices}
    \subsection{Introduction}
    You’ve definitely found yourself opening the last few pages of a textbook,
    and you would’ve noticed a little handy something called the index \index{index}. 
    It contains a list of all the keywords \index{keywords} found in the book, and where to find those keywords 
    in the book. If you look at this document, written in \LaTeX \index{\LaTeX}, you’ll find a similar section right at the end.

    \subsection{Make Index}
    We use the package \textbf{imakeidx} for creating the index. Another package available
    for this is \textbf{makeidx} but it offers less customization.
    The index declaration is done in the preamble, using \verb!\!makeindex.
    Words are added to the index by declaring a tag \verb!\!index\{``word''\}.
    The index is then printed using \verb!\!printindex.

    \subsection{Entries and Subentries}
    In a proper index, most entries also include subentries which denote that 
    the subentry is preceded by the main entry at so and so page.
    For example, the \LaTeX{} keyword \index{\LaTeX!keyword} is nested with the word ``keyword'' in the index of this document
    which means that the keywords ``\LaTeX{} keyword'' come at that page.
    Making an index \index{index} section \index{index!section} is easily done using the 
    \textbf{imakeidx} package.

    \subsection{Index Formatting} 
    Index formatting is easily done by passing optional arguments to \verb!\!makeindex.
    For example, in this document, I have set ``columns'' (not column) argument equal to 1.

    \subsection{Getting index inside the TOC}
    By default, the index is not included inside the Table of Contents. 
    To include it, we add another optional agrument to \verb!\!makeindex ``intoc''
    
\clearpage

\section{Multiple Columns}
    \subsection{Two Column}
    FOr creating a document which seperates the page into two and then writes
    (like a newspaper), just add [twocolumn] parameter to the \verb!\!documentclass\{\}
    like, \verb!\!documentclass[twocolumn]\{article\}.

    \subsection{Multicol}
    For adding multiple columns to the document, or simply for more flexibility in writing columns
    we can take the help of the \textbf{multicol} package. Here I've used the \textbf{lipsum} package as well to generate 
    Lorem Ipsum placeholder text. The multicols is used as a seperate envionment.
    
        \begin{multicols}{3}
        [
            \subsubsection{Multicol sample}
            Whatever is placed inside the square brackets is used as a header text.
            The braces after multicols enviornment declaration decide the number of columns.
        ]
        \lipsum[1]
        \end{multicols}
    
        \subsubsection{Additional Parameters}
        We can set the seperation between the columns, which is defined by \verb!\!columnsep. We can
        set it to whatever length we want by using \verb!\!setlength\{\verb!\!columnsep\}\{1cm\} 
        his affects all the columns present below this command.

        In the default multicols environment the columns are balanced so each one contains the 
        same amount of text. This default format can be changed by the starred environment \textbf{multicols*}

        To break the column we can use \verb!\!columnbreak. However, when a column is broken,
        the remaining words are arranged in such a way to use up all the space present in that column
        and hence, can have unwanted effects.

        \subsubsection{Vertical Rulers}
        The columns can be seperated using vertical lines.To add them, 
        use \verb!\!setlength\{\verb!\!columnseprule\}\{1pt\} in the preamble.
        For changing the color, you can add \verb!\!def\verb!\!columnseprulecolor\{\verb!\!color\{blue\}\}
        in the preamble.









    %running courses


\printindex

\end{document}