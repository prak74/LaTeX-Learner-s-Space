\documentclass{article}

% Use Packages =================================
\usepackage{xcolor}
\usepackage[utf8]{inputenc}
\usepackage{parskip}[parfill]
\usepackage{ragged2e}
\usepackage{multicol}
\usepackage[a4paper, hmargin= 1in, vmargin= 1.5cm, headheight=47pt, headsep=0.5cm]{geometry}
\usepackage{fancyhdr}
\usepackage{hyperref}
\usepackage{tgschola}

% Make Header ==================================
\pagestyle{fancy}
\fancyhf{}
\lhead{Prakhar Mittal \\ B.Tech Electrical Engineering (2019-23) \\ CPI: FF.FF}
\rhead{190070046 \\ UG Second Year \\ 190070046@iitb.ac.in \\ +91 8839365217}

% % Page Color =================================
% \pagecolor{black}
% \color{white}

% Commands
\newcommand{\heading}[1]{\fontfamily{qtm}\selectfont{\textcolor{blue!50!green}{\LARGE{#1}}}\normalsize}     % Create section headings
    
\newcommand{\topic}[1]{\fontfamily{qcs}\selectfont{\large{\textbf{#1}}}\normalsize} % Subsections

\newcommand{\sidetext}[1]{\hfill \small{\textit{\textcolor{black!30}{(#1)}}}\normalsize} % right-text

% Begin ========================================
\begin{document}
\newgeometry{top=3in}

\heading{Projects} 

% \vspace{0.5em}
    \topic{Image to \LaTeX{} Converter} \sidetext{ITSP 2019-20, IIT-Bombay}
    
    \begin{itemize}
        \item Built a Deep Learning model for generating \LaTeX{} expression for a given mathematical expression 
        uploaded in the form of an image.        
        \item Used the Django framework to create a web application that allows the user to input the mathematical expression, 
        and yield a compilable \LaTeX{} code for the same.
        \item Seamlessly converted the equation to \LaTeX{} in order to
        ease documentation for research work/homework/ wherever the user wishes to implement.
        \item Also provided a compiled pdf file to show how the predicted code would look when compiled with a TeX editor, downloadable from the website using djano-tex. 
        \item[Link to Project] \url{https://github.com/hawkeye-ITSP/webpage}
    \end{itemize}

\vspace{0.5em}    
    \topic{Text Sentiment Analysis} \sidetext{ITSP 2019-20, IIT-Bombay}
    
    \begin{itemize}
        \item Created a tool which could classify any sentence on the basis of it's sentiment into positive or negative and help automate classification of feedbacks inside the institute.
        \item Used RNN and NLTK library in python for preprocessing and labelling of sentences.
        \item Learned and implemented Naltural Language Processing algorithms, primarily, to deal with sentences containing words with both positive as well as negative sentiment.
        \item[Link to Project] \url{https://github.com/kenzz17-ITSP/webpage}
    \end{itemize}

\vspace{0.5em}    
    \topic{ML-Gym} \sidetext{SoC 2019-20, IIT-Bombay}
    
    \begin{itemize}
        \item Worked in a team of 6 under a senior mentor to develop a platform where 
        a ML user can see how an algorithm is behaving and can tweak its parameter to see 
        result change in real time.
        \item Manually implemented training and testing for Gaussian-SVM in python and developed a Django-HTML architecture for applying the same.
        \item The website will provide the user with the flexibilty of choosing training algorithms for his/her uploaded dataset as well as to choose and compare accuracy between any hyperparameter combination of their choice for the algorithm.
        \item[Link to Project] \url{https://github.com/codeLAlit/MLGYM}
    \end{itemize}

\vspace{0.8em}
\heading{Technical Skills}

    \begin{description}
        \item [Programming] Python, C++, Bash
        \item [Web Development] HTML, CSS, Bootstrap, JavaScript, Django
        \item [Office Tools] \LaTeX{}, git, AutoCAD, SolidWorks
    \end{description}
        
\restoregeometry
\thispagestyle{empty}

\heading{Scholastic Achievements}

    \begin{itemize}
        \item Secured \textbf{All India Rank 127} (among 245,000) in JEE Advanced, considered one of the toughest
        entrance exams in the world. \sidetext{2019}
        \item Secured \textbf{All India Rank 121} in JEE Mains out of more than 1 million candidates. \sidetext{2019}
        \item Scored \textbf{438 marks} out of 450 in \textbf{BITSAT} \sidetext{2019}
        \item Awarded scholarship in the \textbf{KVPY Fellowship Programme}, given only to around 1500 students throughout the country.\sidetext{2019}
        \item Qualified for \textbf{INPHO}, Indian National Physics Olympiad, having placed in the \textbf{Top 1\%} nationwide, out of 50,000 candidates in \textbf{NSEP}, National Standard Examination in Physics. \sidetext{2018}
        \item Qualified for \textbf{INCHO}, Indian National Chemistry Olympiad, having placed in the \textbf{Top 1\%} nationwide, out of 70,000 candidates in \textbf{NSEC}, National Standard Examination in Chemistry. \sidetext{2018}
    \end{itemize}

\vspace{0.8em}
\heading{Co-Curricular Courses}
    \begin{itemize}
        \item \textbf{Machine Learning} MOOC by Andrew NG on coursera.org \sidetext{Summer,2020}
        \item \textbf{Deep Learning} MOOC by deeplearning.ai on coursera.org \sidetext{Summer,2020}
        \item \textbf{Basics of Web Development} from codecademy.com \sidetext{Summer,2020}
        \item \textbf{Introduction to \LaTeX{}} by Learner's Space, IIT Bombay \sidetext{Summer,2020}
    \end{itemize}

\vspace{0.8em}
\heading{Course Projects} 
    
    \topic{Voltage Source Design} \sidetext{Joseph John, EE-113 Introduction to Electrical Engineering Practice}
    \begin{itemize}
        \item Worked as a team of two to create a Regulated Voltage Supplier using basic knowledge about transformers, ICs, diodes and basic electrical elements.
        \item The device would take in an AC supply from any plug point, and output a regulated supply of two constant voltage supplies, 5V and 12V.
        \item Used resistors, capacitors, diodes and ICs to design a suitable circuit and realized it on a PCB.
    \end{itemize}

    \topic{Basics of Control Systems} \sidetext{Debraj Chakraborty, EE-113 Introduction to Electrical Engineering Practice}
    \begin{itemize}
        \item Engineered a linear control system pipeline in MATLAB Simulink for instant motion and alignment of a rod (imitating the role of a satellite dish) in the desired direction.
        \item Also studied the effect of various starter signals, and feedback loops, optimized them, ultimately leading to the intuitive understanding of PID controllers.
    \end{itemize}

\vspace{0.8em}
\heading{Extra Curricular Activities} 

    \begin{itemize}
        \item Engineered a \textbf{Remote Controlled bot} capable of negotiating different obstacles through a 100m long track. Stood \textbf{6th} out of more than 100 other teams.
        \item Successfully completed a year long \textbf{NSO} programme in Lawn Tennis, organised by IIT Bombay 
    \end{itemize}
\end{document}