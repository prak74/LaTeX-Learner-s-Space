\documentclass{knittingpattern}

% Use Packages ========================================
\usepackage{xcolor}
\usepackage{lipsum}
\usepackage{skak}
\usepackage{url}
\usepackage{hyperref}
 
% Title ===============================================
\title{FIeld Specific \LaTeX{} tricks}
\author{Prakhar Mittal}

% Main Body ===========================================
\begin{document}

\maketitle
\tableofcontents

\newpage

\section{Knitting Patterns}
This class provides a very convenient way to introduce boxed diagrams. We are thus going to use
our stock image a few more times. Also, it has few features to make knitting instructions more readable,
however, we can adapt them to make prettier documents for our purposes as well.
\diagram{graph.png}

\important[0.7]{black}{blue!40}{We have a way of highlighting important text, or as was originally intended,
important instructions. Feel free to choose whatever background and border colour you like when you replicate
these features, but try to replaicate the dimensions as well as you can}

\begin{pattern}[0.2]{green!50!}{red!20}
    \textbf{Course} & \textbf{Credits} \\
        Introduction to Computer Programming & 6 \\
        Abstractions and Paradigms in Programming & 6 \\
        Abstractions and Paradigms in Programming Lab & 3 \\
        Data Structures and Algorithms & 6 \\
        Softwares Systems Lab & 8 \\
\end{pattern}

\begin{note}{black}{blue!20!green!30}{Note}{
    This is a note. The above feature was introduced to typeset a sequence of knitting instructions.
    The first column is for the instruction, the second for the number of stitches. But hey, it looks
    cool!
}
\end{note}

\biog[0.4]{graph.png}{\lipsum[1] Download knittingpattern \href{https://ctan.org/pkg/knittingpattern?lang=en}{from here}.}

\clearpage

\newgame









\end{document}